
\documentclass[10pt]{article}
\usepackage{amssymb,amsmath,amscd}
\usepackage{hyperref}

\input epsf.tex


\title{Understanding clustering of textual documents}



\begin{document}
\maketitle

\newcommand{\op}[1]{\operatorname{#1}}


\newcommand{\CC}{{\mathcal C}}

\renewcommand{\O}{{\mathcal O}}

\newcommand{\E}{{\mathcal E}}
\newcommand{\F}{{\mathcal F}}

\newcommand{\g}{{\mathfrak g}}
\newcommand{\h}{{\mathfrak h}}

\renewcommand{\k}{{\bf k}}
\newcommand{\kk}{{\overline{\bf k}}}





\newtheorem{thm}{Theorem}[section]
\newtheorem{conj}[thm]{Conjecture}
\newtheorem{cor}[thm]{Corollary}
\newtheorem{prop}[thm]{Proposition}
\newtheorem{lemma}[thm]{Lemma}
\newenvironment{pf}{\paragraph{Proof}}{\qed\par\medskip}
\newtheorem{defn}[thm]{Definition}
\newtheorem{formula}[thm]{Formula}
\newtheorem{thm*}[thm]{Theorem$^*$}
\newtheorem{remark}[thm]{Remark}
\newtheorem{remarks}[thm]{Remarks}
\newtheorem{example}[thm]{Example}
\newtheorem{examples}[thm]{Examples}\newtheorem{ass}[thm]{Assumptions}
\newcommand {\A}{\mathcal A}

%\newcommand{\pf}{\noindent{\sc Proof. }}
\newcommand {\halmos}{\hfill{$\blacksquare$}}
\newcommand{\D}{\operatorname{D}}
\newcommand{\diam}{\diamond}
\newcommand {\C}{\mathbb C}
\newcommand {\ol}[1]{\overline{#1}}
\newcommand {\IF}{\mathbb{F}}
\newcommand {\Hom}{\operatorname{Hom}}
\newcommand {\Aut}{\operatorname{Aut}}
\newcommand {\Ext}{\operatorname{Ext}}
\newcommand {\Rep}{\operatorname{Rep}}
\renewcommand {\H}{\operatorname{\mathcal H}}
\newcommand{\Vect}{\operatorname{Vect}}
\newcommand {\id}{\operatorname{id}}
%\newcommand {\cplx}[4]{#1\rightleftarrows^{#2}_{#3}#4}
\newcommand{\cplx}[4] {\begin{xy}
\xymatrix@=0pt@C=20pt@R=15pt{#1
\ar@<.5ex>[r]^{#2}		& \ar@<.5ex>[l]^{#3}  #4}
\end{xy}  }
\newcommand{\mat}[4]{\begin{pmatrix}#1&#2\\#3&#4\end{pmatrix}}



\newcommand{\cl}[1]{{\hat{#1}}}

\newcommand{\onto}{\twoheadrightarrow}
\newcommand{\gen}[1]{[#1]}
\newcommand{\R}{{\operatorname{R}}}
\newcommand{\lra}{\longrightarrow}

\newcommand {\aand}{\qquad\text{and}\qquad}
\newcommand {\comment}[1]{\footnote{#1}}
\newcommand {\<}{\langle}
\renewcommand {\>}{\rangle}
\newcommand {\Omit}[1]{}
\newcommand {\IP}{\mathbb{P}}
\newcommand{\isom}{\cong}
\newcommand{\half}{\frac{1}{2}}
\newcommand{\tensor}{\otimes}
\renewcommand{\dag}{*}
\newcommand{\fb}{\mathfrak{b}}
\newcommand{\e}{\operatorname{e}}

\newcommand{\Zt}{{\mathbb{Z}_2}}
\newcommand{\Ho}{\operatorname{Ho}}
\renewcommand{\P}{\mathcal{P}}
\newcommand{\Z}{\mathbb{Z}}
\newcommand{\red}{\operatorname{red}}
\newcommand{\n}{\mathfrak{n}}
\newcommand{\blank}{\kern.0mm\smash{-}\kern.0mm}
\newcommand{\blob}{{\scriptscriptstyle\bullet}}
\newcommand{\B}{\mathcal{B}}
\newcommand{\K}{K}
\newcommand{\into}{\hookrightarrow}
\newcommand{\longinto}{\lhook\joinrel\longrightarrow}
\newcommand{\rad}{\operatorname{rad}}
\newcommand{\lRa}[1]{\stackrel{#1}{\lra}}

\newcommand{\tw}{\operatorname{tw}}
\newcommand{\Htw}{\H_{\tw}}
%\newcommand{\red}{\operatorname{red}}
\renewcommand{\DH}{\operatorname{\mathcal{DH}}}
\newcommand{\DHred}{\operatorname{\mathcal{DH}_{red}}}
\newcommand{\U}{\operatorname{U}}
\newcommand{\im}{\operatorname{im}}
\newcommand{\coker}{\operatorname{coker}}
\newcommand{\Iso}{\operatorname{Iso}}



\tableofcontents


	\section*{Abstract} 
	\section{Introduction}
	\section{Gap statistics}
	\section{Distribution for tf-idf vectors}
	To compute the gap statistic, we need a random sample from the distribution with no clusters. Now we would want this distibution to be similar of the distribution of tf-idf vectors. It has been argued\cite{MKE} that the Dirichlet compound multinomial (DCM) distribution models the bag-of-words in textual documents. As the name suggests, DCM is a compound of Dirichlet and multinomial distributions. We will make this notion clear in the following. Before we delve into DCM, we will talk about multinomal and Dirichlet distributions. 


\subsection{Multinomial Model}
Multinomial distribution is a multivariate generalization of the binomial distribution. It models the counts of each of the p outcomes in n trials. e.g. consider the experiment of throwing a dice n times, this has 6 possible outcomes. A sample drawn from $Mult(n,6)$ is of the form $(x_1, x_2, \ldots, x_6)$ where each $x_i$ is the count of number of times the outcome was i. 
As is pretty evident from the description that the count vectors for words in a document can be modelled by such a distribution. More specifically, a multinomial distribution can be used to model the probability of observing a given vector of word counts in a document. If $\theta_w$ is the probability of generating a word $w$, the the prob of generating a document $x$ can be expressed as - 
\begin{equation}
p(x|\theta) = \frac{n!}{\prod_{w=1}^W x_w!} \prod_{w=1}^W \theta_w^{x_w}
\end{equation}
where $\theta = (\theta_1, \theta_2, \ldots, \theta_W)$, $x_w$ is the frequency of word $w$ in the document, $W$ is the size of the vocabulary, and $n=\sum x_w$.\\
$\theta$ is the parameter of the distribution, which can be approximated as follows\cite{MKE}:
\begin{equation}
\hat{\theta_w} = \frac{\sum_(d=1)^D x_{dw}}{\sum_w=1^W x_{dw}}
\end{equation}
where $d$ is the document number and $D$ is the total number of documents in the corpus. 

\subsection{Dirichlet Model}
Dirichlet distribution is a continous distribution which is the multivariate generalization of beta distribution. It is parameterized by a vector $\alpha$ containing real numbers. The Dirichlet distribution has the probability density function - 
\begin{equation}
p(\theta | \alpha) = \frac{\Gamma(\sum_{w=1}^W \alpha_w)}{\prod_{w=1}^W \Gamma(\alpha_w)} \prod_{w=1}^W \theta_w^{\alpha_w - 1}
\end{equation}
Dirichlet distribution can be used to model probabilities i.e. as a distributions over distributions. Thus $\theta$ is the above expression is a vector of probabilities.

\subsection{Dirichlet Compound Multinomial (DCM) model}
A DCM model of textual documents can be constructed hierarchically, exploiting the fact that Dirichlet distribution can describe probabilities. {\em In DCM model, the word-count vector for each document can be generated by a multinomial distribution whose parameters are generated by the Dirichlet distribution.} Thus to generate a sample from DCM,
\begin{enumerate}
\item Draw a sample from the Dirchlet distribution with parameter $\alpha$. This gives $\theta$.
\item With this $\theta$ as parameter, get a sample from multinomial distribution to get the count vector $x=(x_w)$.
\end{enumerate}
This gives us word-count vectors, to generate sample representing tf-idf vectors, we apply appropriate transformation to the vectors thus generated. \\ 
We followed the above steps to generate samples of vectors which are well-suited for the cluster-analysis. \\
Mathematically, for DCM model
\begin{equation}
p(x| \alpha) = \int_{\theta} p(x| \theta) p(\theta | \alpha) d\theta
\end{equation}
The only parameter of the DCM model is $\alpha$ which can be approximated as 
\begin{equation}
\hat{\alpha_w} \simeq \frac{1}{D} \sum_{d=1}^D I(x_{dw} > 1)
\end{equation}
Since $x_{dw} = 0$ for most of the words in the vocabulary, $\alpha_w << 1$ for most words. Experiments involving training on popular data sets show the value to be less that 0.1.

	\section{Computation of Gap statistics}
	\section{Conlusion}
	

%%%%%%%%%%%%%%%%%%%%%%%%%%%%%%%%%%%%%%%%%%%%%%%%%%%%%%%%%%%%%%%%%%%%%%%%%%%%%%%%%%%%%%%%%%%%%%
%%%%%%%%%%%%%%%%%%%%%%%%%%%%%%%%%%%%%%%%%%%%%%%%%%%
\bibliographystyle{amsplain}
\begin{thebibliography}{100}

\bibitem{MKE} Rasmus E. Madsen, David Kauchak, Charles Elkan, Modeling Word Burstiness Using the Dirichlet Distribution, Proceedings of the 22 nd International Conference on Machine Learning, Bonn, Germany, 2005 \href{http://www.machinelearning.org/proceedings/icml2005/papers/069_WordBursting_MadsenEtAl.pdf}{(Online version)}
\bibitem{Elkan} Charles Elkan, Deriving TF-IDF as a Fisher Kernel, M. Consens and G. Navarro (Eds.): SPIRE 2005, LNCS 3772, pp. 296–301, 2005. \href{http://cseweb.ucsd.edu/~elkan/papers/spire05.pdf}{(Online version)}




\end{thebibliography}




\end{document}
